\chapter{Basic security notions}
\section{Definition}
The theory of network security intersecates with many other fields:

\begin{itemize}
  \item theory of \textbf{computational complexity} (to state what
  \textit{feasible} solutions are)
\item \textbf{information theory} (provides fundamental \textit{limits} of network
  security)
\item \textbf{game theory} (helps to identify optimal \textit{strategies})
\end{itemize}

The field of network security is a subset of a more extended field, which is
\textbf{information security}. The scope of network security is limiteed to
only those situations in which a \textit{communication between peers} is
involved.\\
Note, however, that boundaries between information security's subfields are
rather loose.

\section{Taxonomy: objectives, services, attacks, threats and mechanisms}
To explain what network security is and what its objectives are, we can start
from NIST's definition of \textit{\textbf{computer security}}:

\begin{quote}
  The protection afforded to an automated information system in order to attain
  the applicable objectives of preserving integrity, availability and
  confidentiality of information system resources.
\end{quote}

From this definition we can then identify a list of objectives that apply to
network security:

\begin{itemize}
  \item \textbf{Confidentiality}: information is not made available to unauthorized
    individuals
  \item \textbf{Integrity}: information is changed only in a specific and authorized way
  \item \textbf{Availability}: communication must not be easily disrupted (different
    from the concept of reliability)
  \item \textbf{Accountability}: requirement of being able to know who's responsible for
    each and every action or event in the network
  \item \textbf{Privacy / Anonimity}: assures that individuals can control or influence
    what information related to them may be collected and stored and by whom
    and to whom that information may be disclosed
\end{itemize}

Strictly related to the concept of network security are the concepts of
\textit{threat} and \textit{attack}:

\begin{itemize}
  \item \textbf{Threat}: a potential for violation of security. That is, a
    \textit{possible} danger that may exploit a vulnerability
  \item \textbf{Attack}: an assault on system security that derives from an
    intelligent threat; that is, an intelligent act that is a
    \textit{deliberate attempt} to evade security services
\end{itemize}

In order to aid managers in the task of organizing and providing security,
\textbf{OSI security architecture} (ITU-T Recommendation X.800) provides three
distinct but related concepts:

\begin{itemize}
  \item security \textbf{attack}: any action that compromises security; that
    is, actions aimed ad disrupting security objectives
  \item security \textbf{mechanism}: a process that is designed to detect,
    prevent or recover from a security attack
  \item security \textbf{service}: a processing or communication service that
    enhances the security of data transfers. Security services aim to counter
    security attacks and make use of one or more security mechanisms
\end{itemize}

We can identify a list of common security attacks, services and mechanisms.

\subsection*{Security threats/attacks}
\begin{itemize}
  \item eavesdropping (against confidentiality)
  \item modification (against integrity)
  \item denial of service (against availability)
  \item forgery (against accountability)
  \item repudiation (either from sender or from receiver) (against accountability)
  \item profiling/fingerprinting (against privacy)
\end{itemize}

\subsection*{Security services}
\begin{itemize}
  \item secrecy (or confidentiality)
  \item data integrity
  \item access control
  \item message authentication
  \item notarization
  \item anonimization
\end{itemize}

\subsection*{Security mechanisms}
\begin{itemize}
  \item encryption (can provide secrecy)
  \item digital signature (can provide integrity, message authentication and
    notarization)
  \item intrusion detection, authentication and key distribution (can provide
    access control)
\end{itemize}

\section{Positioning of security services in network}
\textit{ITU-T Recommendation X.800} defines some general security-related architectural
elements which can be applied when protection of communication is a concern.\\
The recommendation extends the traditional \textit{OSI Reference Model
(Recommendation X.200} by specifying which services each layer is supposed to
support, and which mechanisms should be used to carry out each of those
services.\\
It should be noted that provision of any security service is optional and
depends upon requirements. Also, layers may not always provide the security
services from within themselves, but may make use of appropriate security
services being provided at lower layers. Even when no security services are
being provided within a layer, the service definitions of that layer may
require modification to permit requests for security services to be passed to a
lower layer.

\section{The probabilistic success/complexity trade-off} 
\section{Computational security}
\section{Unconditional security}
