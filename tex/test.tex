\chapter{Test}
\section{Sample paragraph}
The theory of network security intersecates with many other fields:

\begin{itemize}
  \item theory of \textbf{computational complexity} (to state what
  \textit{feasible} solutions are)
  \item information theory (provides fundamental \textit{limits} of network
  security)
  \item game theory (helps to identify optimal strategies)
\end{itemize}

\subsection{Sample subsection}
Trying to visualize \texttt{monospace\_words}. Also inline formulas like this:
$k_j$.

This should be rendered as a code snippet:
\begin{lstlisting}[language=Matlab]
for i = 1:very_large_number
  msg = dec2hex(randi([0,2^32-1],1,1))
  assert(isequal(
    msg,
    feistel_decrypt(
      feistel_encrypt(msg, key, 4, ...
        @linear_round_function, @half_outward_shift), 
      key, 4, @linear_round_function, @half_outward_shift),
    'u and u_hat are supposed to be equal'))
end
\end{lstlisting}

\subsubsection{Sample subsubsection}
Another level of the hierarchy.
\begin{verbatim}
  3C07E881
  3C07177E
  C3F8E881
\end{verbatim}

\section{Sample formula}
$\Psi_{-1}(n) = e^{-i\omega_0n} = \Psi_{p-1}(n) = e^{i(p-1)\omega_0n }$

